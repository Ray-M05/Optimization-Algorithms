\documentclass[11pt]{article}
\usepackage[spanish]{babel}
\usepackage[utf8]{inputenc}
\usepackage[T1]{fontenc}
\usepackage{geometry}
\usepackage{amsmath, amssymb, amsthm, mathtools}
\usepackage{enumitem}
\usepackage{hyperref}
\usepackage{siunitx}
\usepackage{booktabs}
\usepackage{xcolor}
\geometry{margin=1in}
\setlength{\parskip}{6pt}
\setlength{\parindent}{0pt}

\title{\textbf{Tarea evaluativa de Optimización}\\[2mm]
\large C312}
\author{Rachel Mojena González}

\begin{document}
\maketitle

\section{Formulación del modelo}
Se considera la función objetivo
\begin{equation}\label{eq:obj}
f(x,y)=\cos x\,\sin y-\frac{x}{y^2+1},\qquad (x,y)\in\mathbb{R}^2,
\end{equation}
la cual es $C^\infty$ en todo el plano. Dado que $f(x,0)=-x\to -\infty$ cuando $x\to +\infty$, el problema irrestricto no está acotado inferiormente y carece de mínimo global. Para efectos de análisis y verificación con KKT, se trabaja la versión acotada
\begin{equation*}
\mathcal{B}=[-100,100]^2=\{(x,y): -100\le x\le 100,\ -100\le y\le 100\}.
\end{equation*}
En forma estándar, las restricciones se escriben como
\begin{equation}\label{eq:res}
\begin{aligned}
g_1(x,y)&=x-100\le 0, &
g_2(x,y)&=-x-100\le 0,\\
g_3(x,y)&=y-100\le 0, &
g_4(x,y)&=-y-100\le 0.
\end{aligned}
\end{equation}

\section{Diagnóstico teórico}
\subsection{Gradiente y hessiana}
Las derivadas parciales de primer orden son
\begin{equation}\label{eq:grad}
\begin{aligned}
f_x(x,y) &= -\sin x\,\sin y-\frac{1}{y^2+1},\\
f_y(x,y) &= \cos x\,\cos y+\frac{2xy}{(y^2+1)^2}.
\end{aligned}
\end{equation}
La matriz hessiana resulta
\begin{equation}\label{eq:hess}
\nabla^2 f(x,y)=
\begin{bmatrix}
-\cos x\,\sin y & -\sin x\,\cos y+\dfrac{2y}{(y^2+1)^2}\\[6pt]
-\sin x\,\cos y+\dfrac{2y}{(y^2+1)^2} & -\cos x\,\sin y+\dfrac{2x(1-3y^2)}{(y^2+1)^3}
\end{bmatrix}.
\end{equation}

\subsection{No convexidad}
Para que $f$ fuera convexa en $\mathbb{R}^2$ sería necesario que su hessiana fuese semidefinida positiva (SDP) para todo $(x,y)$.

Evalúese en el punto $(x_0,y_0)=\big(0,\tfrac{\pi}{2}\big)$, donde $\sin y_0=1$ y $\cos y_0=0$:
\begin{equation*}
\nabla^2 f\big(0,\tfrac{\pi}{2}\big)=
\begin{bmatrix}
-1 & \dfrac{2(\tfrac{\pi}{2})}{\big((\tfrac{\pi}{2})^2+1\big)^2}\\[8pt]
\dfrac{2(\tfrac{\pi}{2})}{\big((\tfrac{\pi}{2})^2+1\big)^2} & -1
\end{bmatrix}
=
\begin{bmatrix}
-1 & \alpha\\ \alpha & -1
\end{bmatrix},
\qquad
\alpha=\frac{\pi}{\left((\tfrac{\pi}{2})^2+1\right)^2}\approx 0.261.
\end{equation*}
Para matrices simétricas $2\times 2$, la condición SDP es equivalente a $a\ge 0$ y $\det\ge 0$ para $H=\begin{bmatrix}a&b\\b&c\end{bmatrix}$. Aquí $a=-1<0$, luego $H$ no es SDP. De hecho, $\det(H)=(-1)^2-\alpha^2>0$ y ambos autovalores son negativos ($-1\pm\alpha$), con lo cual la hessiana es definida negativa en ese punto. Por consiguiente, $f$ no es convexa.

\subsection{Puntos críticos del problema irrestricto}
Los puntos críticos interiores satisfacen $\nabla f=0$, esto es
\begin{equation}\label{eq:crit}
\boxed{\sin x\,\sin y=-\frac{1}{y^2+1}},\qquad
\boxed{\cos x\,\cos y=-\frac{2xy}{(y^2+1)^2}}.
\end{equation}
Debido a la periodicidad de las funciones trigonométricas, existen infinitos candidatos en $\mathbb{R}^2$ y, en particular, dentro de $\mathcal{B}$.

\section{Condiciones KKT en \texorpdfstring{$\mathcal{B}$}{[-100,100]^2}}
Sea
\begin{equation*}
\mathcal{L}(x,y,\lambda)=f(x,y)+\sum_{i=1}^4 \lambda_i g_i(x,y),\qquad \lambda_i\ge 0.
\end{equation*}
Las KKT necesarias bajo LICQ, que se satisface pues los gradientes activos son afines e independientes.

El análisis se realiza por \emph{conjuntos activos} $I(x,y)=\{i:\ g_i(x,y)=0\}$.

\subsection*{Interior $I=\varnothing$}
Del problema irrestricto se obtiene $\lambda_i=0$ y $f_x=0$, $f_y=0$. Se listan numéricamente algunos candidatos interiores dentro de $\mathcal{B}$, por ejemplo
$(x,y)\approx (0.7946,-0.8935)$ con $f\approx -0.9878$ y
$(x,y)\approx (4.4881,-3.2314)$ con $f\approx -0.4122$.

\subsection*{Borde derecho $I=\{1\}$ ( $x=100$ )}
Las ecuaciones KKT quedan
\begin{equation*}
\cos(100)\cos y+\frac{200\,y}{(y^2+1)^2}=0,\qquad
\lambda_1=-f_x(100,y)=\sin(100)\sin y+\frac{1}{y^2+1}\ \ (\ge 0).
\end{equation*}
Entre las soluciones en $[-100,100]$, el valor
\begin{equation*}
y\*\approx -0.0043117
\end{equation*}
satisface $\lambda_1\approx 1.0022\ge 0$ y produce
\begin{equation*}
f(100,y\*)\approx -100.001859.
\end{equation*}

\subsection*{Borde izquierdo $I=\{2\}$ ( $x=-100$ )}
Se obtiene
\begin{equation*}
\cos(100)\cos y-\frac{200\,y}{(y^2+1)^2}=0,\qquad
\lambda_2=f_x(-100,y)=\sin(100)\sin y-\frac{1}{y^2+1}\ \ (\ge 0).
\end{equation*}
Los candidatos resultantes presentan valores significativamente superiores (en valor absoluto menores) al del caso $x=100$; por ejemplo $f(-100,-39.2737)\approx -0.7975$.

\subsection*{Borde superior $I=\{3\}$ ( $y=100$ )}
Las KKT imponen
\begin{equation*}
-\sin x\,\sin(100)-\frac{1}{100^2+1}=0,\qquad
\lambda_3=-f_y(x,100)\ \ (\ge 0).
\end{equation*}
Entre las soluciones factibles en $[-100,100]$, el mejor valor observado es $f(x,100)\approx 0.4966$ en $x\approx 97.3892$.

\subsection*{Borde inferior $I=\{4\}$ ( $y=-100$ )}
Análogamente,
\begin{equation*}
-\sin x\,\sin(-100)-\frac{1}{100^2+1}=0,\qquad
\lambda_4=f_y(x,-100)\ \ (\ge 0),
\end{equation*}
con mejores valores alrededor de $0.497$, lejos del caso $x=100$.

\subsection*{Esquinas $I=\{1,3\},\{1,4\},\{2,3\},\{2,4\}$}
En las cuatro combinaciones, algún multiplicador resulta negativo, por lo que no satisfacen KKT; además, sus valores de $f$ no son competitivos.

\paragraph{Conclusión en  $\mathcal{B}$.}
Por comparación de los candidatos KKT dentro de $\mathcal{B}$ se obtiene
\begin{equation*}
(x,y)=\big(100,\,y\big),\qquad y\approx -0.0043117,\qquad \lambda_1\approx 1.0022,\quad \lambda_{2,3,4}=0,
\end{equation*}
con valor
\(
f(x\*,y\*)\approx -100.001859.
\)
Dado que el dominio es compacto y $f$ es continua, este punto corresponde al mínimo global del problema acotado.

\section*{Anexo: Algoritmos}
En el archivo jupyter notebook se plasman los algoritmos utilizados: máximo descenso y Newton.
\end{document}